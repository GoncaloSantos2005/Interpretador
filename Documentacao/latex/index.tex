\chapter{Projeto de Sistemas Operativos -\/ Interpretador de Linha de Comandos }
\hypertarget{index}{}\label{index}\index{Projeto de Sistemas Operativos -\/ Interpretador de Linha de Comandos@{Projeto de Sistemas Operativos -\/ Interpretador de Linha de Comandos}}
Este projeto tem como objetivo desenvolver um interpretador de linha de comandos simples, capaz de executar comandos personalizados (como {\ttfamily mostra} e {\ttfamily lista}) e comandos do sistema Unix/\+Linux.

O interpretador funciona em ciclo contínuo até ser introduzido o comando {\ttfamily termina}. É responsável por interpretar a linha de entrada, separar argumentos, tratar comandos inválidos, executar processos (com fork e execvp) e reportar o código de saída de cada comando.

Funcionalidades principais\+:
\begin{DoxyItemize}
\item Execução de comandos personalizados
\item Execução de comandos do sistema
\item Gestão de processos com fork e waitpid
\item Tratamento de erros e mensagens de ajuda
\item Separação modular do código para facilitar manutenção e extensibilidade
\end{DoxyItemize}

O projeto foi desenvolvido no contexto da unidade curricular de Sistemas Operativos do curso de Engenharia de Sistemas Informáticos no Instituto Politécnico do Cávado e do Ave (IPCA).

\begin{DoxyAuthor}{Author}
Gonçalo Santos e Rodrigo Cruz
\end{DoxyAuthor}
\begin{DoxyDate}{Date}
Maio de 2025 
\end{DoxyDate}
